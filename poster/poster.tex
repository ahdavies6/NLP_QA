\documentclass[a1,landscape]{a0poster}
% \usepackage[margin={5cm,1cm,5cm,5cm}]{geometry}
\usepackage[left=5cm, right=5cm, top=7cm, bottom=2cm]{geometry}
\setlength\textwidth{78cm}
\usepackage[compact]{titlesec}

\usepackage{multicol} % This is so we can have multiple columns of text side-by-side
\columnsep=70pt % This is the amount of white space between the columns in the poster
\columnseprule=3pt % This is the thickness of the black line between the columns in the poster

\usepackage[svgnames]{xcolor} 

\usepackage{times} 

\usepackage{graphicx} % Required for including images
\graphicspath{{figures/}} % Location of the graphics files
\usepackage[font=small,labelfont=bf]{caption} % Required for specifying captions to tables and figures
\usepackage{amsfonts, amsmath, amsthm, amssymb} % For math fonts, symbols and environments
\usepackage{wrapfig} % Allows wrapping text around tables and figures

\begin{document}

\begin{minipage}[c]{0.9\linewidth}
\centering
\Huge \color{NavyBlue} \textbf{Question Answering with 2-Stage Candidate Selection} \color{Black}\\ % Title
\LARGE \textbf{Adam Davies \& Carlos Jimenez}\\ % Author(s)
\end{minipage}

\vspace{1cm} % A bit of extra whitespace between the header and poster content

%----------------------------------------------------------------------------------------

\begin{multicols}{3} % This is how many columns your poster will be broken into, a poster with many figures may benefit from less columns whereas a text-heavy poster benefits from more
\Large

\color{SaddleBrown} % SaddleBrown color for the introduction

\section*{\LARGE Introduction}
\textbf{Our objective} was to design a \textbf{single-document based question answering system}.\\
Our system follows a 2-stage approach. First, sentence selection, followed by answer identification, inspired by the approach outlined by Wang, Liu, Xiao et al.[1]. We used classification, lexical analysis, and a rule-based approach in the first stage, and argument, and semantic analysis in the second stage.

\color{DarkSlateGray} % DarkSlateGray color for the rest of the content

\section*{\LARGE Tools Used}
\textbf{Natural Language Tool-Kit (NLTK)} - The backbone of our text processing, and was integral to lexical analysis. \\
\textbf{Stanford CoreNLP} - The primary parser used for constituency parsing.\\
\textbf{Wordnet} - Wordnet was used to analyze similarity of sentences and arguments. \\
\textbf{SpaCy} - 

%----------------------------------------------------------------------------------------
%	MATERIALS AND METHODS
%----------------------------------------------------------------------------------------

\section*{\LARGE First Stage - Sentence Selection}
Our first stage required question classification into questions types. This enabled us to treat each case differently, and additionally deal with particular sub-cases. \\
For example, if we know we are dealing with a how-much question, we can often identify candidate sentences by those containing MONEY patterns. After narrowing down the pool of candidates, we filter our candidate pool through more general sentence selection subprocesses. \\
\textbf{The ``general" case} - Our system has two generalized processes to rank candidate sentences.\\ For example, one process involved extracting sentences' lemmatized words, while preserving POS tags, and comparing them with those words in the question processed the same way. Sentences are then ranked by similarity, and passed on to the second stage (answer identification).
%------------------------------------------------

\section*{\LARGE Second Stage - Answer Candidate Selection}

Nulla vel nisl sed mauris auctor mollis non sed. 

Curabitur mi sem, pulvinar quis aliquam rutrum. (1) edf (2)
, $\Omega=[-1,1]^3$, maecenas leo est, ornare at. $z=-1$ edf $z=1$ sed interdum felis dapibus sem. $x$ set $y$ ytruem. 

%----------------------------------------------------------------------------------------
%	CONCLUSIONS
%----------------------------------------------------------------------------------------

\color{SaddleBrown} % SaddleBrown color for the conclusions to make them stand out

\section*{\LARGE Conclusions}
\textbf{What Worked Well}
\begin{itemize}
    \item Heuristics and a rule-based approach were hard to abandon.
    \item 
\end{itemize}
\vspace{0.7em}
\textbf{What Did Not}
\begin{itemize}
    \item 
\end{itemize}

\color{DarkSlateGray} % Set the color back to DarkSlateGray for the rest of the content

 %----------------------------------------------------------------------------------------
%	REFERENCES
%----------------------------------------------------------------------------------------

\nocite{*} % Print all references regardless of whether they were cited in the poster or not
\bibliographystyle{plain} % Plain referencing style
\bibliography{sample} % Use the example bibliography file sample.bib

\end{multicols}
\end{document}